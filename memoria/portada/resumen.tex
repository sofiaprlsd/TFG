\cleardoublepage

\chapter*{Resumen\markboth{Resumen}{Resumen}}

El ictus es una de las principales causas de discapacidad motora, afectando a la movilidad y calidad de vida de los pacientes.
La rehabilitación temprana, durante los primeros meses tras el ictus, es crucial para optimizar la recuperación funcional.\\

En los últimos años, diversos estudios han evaluado la efectividad de la integración de dispositivos robóticos en los procesos de rehabilitación, mostrando resultados prometedores en cuanto a la mejora de la movilidad y participación del paciente.\\

El presente trabajo tiene como objetivo desarrollar un sistema de rehabilitación motora para la extremidad superior de pacientes post-ictus, basado en un entorno gamificado personalizado que potencie la motivación del paciente y mejore la efectividad del tratamiento.
La plataforma consta de tres componentes principales: una interfaz de control que permite al terapeuta adaptar las sesiones de rehabilitación de acuerdo con las capacidades del paciente, una interfaz de visualización que facilita el monitoreo en tiempo real del progreso del paciente, y una interfaz de juego que permite hacer los ejercicios de rehabilitación de manera interactiva y dinámica.
Además, el sistema almacena los datos más relevantes durante la terapia, lo que facilita realizar un análisis detallado y evaluar el progreso de cada paciente.\\

Se espera que este sistema no solo facilite la rehabilitación motora, sino que también mejore la motivación y adherencia al tratamiento, ofreciendo una solución más efectiva para la recuperación post-ictus, gracias a su adaptación personalizada a las necesidades de cada paciente, un mayor control sobre los ejercicios y la escalabilidad que permite ajustarse a diferentes modos de terapia.
