\cleardoublepage

\chapter*{Abstract\markboth{Abstract}{Abstract}}

Stroke is one of the main causes of motor disability, affecting patients' mobility and quality of life.
Early rehabilitation, during the first months after stroke, is crucial to optimize functional recovery.\\

In recent years, several studies have evaluated the effectiveness of integrating robotic devices in rehabilitation processes, showing promising results in terms of improved patient mobility and participation.\\

The present work aims to develop a motor rehabilitation system for the upper extremity of post-stroke patients, based on a personalized gamified environment that enhances patient motivation and improves treatment effectiveness.
The platform consists of three main components: a control interface that allows the therapist to adapt the rehabilitation sessions according to the patient's capabilities, a visualization interface that facilitates real-time monitoring of the patient's progress, and a game interface that allows the rehabilitation exercises to be performed interactively and dynamically.
In addition, the system stores the most relevant data during therapy, which facilitates detailed analysis and evaluation of each patient's progress.\\

It is expected that this system will not only facilitate motor rehabilitation, but also improve motivation and adherence to treatment, offering a more effective solution for post-stroke recovery, thanks to its personalized adaptation to the needs of each patient, greater control over the exercises and scalability that allows it to adjust to different modes of therapy.
