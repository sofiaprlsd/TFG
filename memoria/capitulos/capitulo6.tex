\chapter{Conclusiones}
\label{cap:capitulo5}

En este último capítulo se detallan los objetivos y requisitos cumplidos, se redactan las conclusiones que se han obtenido a lo largo de todo el proyecto y se describen las posibles líneas futuras que dan continuidad a este trabajo de fin de grado.\\

Se ha alcanzado el objetivo principal de este proyecto, esto es, diseñar una interfaz gráfica intuitiva y funcional para que un terapeuta controle y personalice las sesiones terapéuticas post-ictus, así como desarrollar un entorno gamificado que motive y facilite la participación activa del paciente durante su rehabilitación motora unilateral.
A su vez, se han cumplido todos los subobjetivos definidos en la Sección \ref{sec:descripcion} y los requisitos establecidos en la Sección \ref{sec:requisitos}:
\begin{enumerate}
    \item Se ha desarrollado un entorno gamificado personalizado al perfil de cada paciente, que incluye diferentes estímulos visuales y auditivos, niveles de dificultad y asistencia, y misiones particulares.
    \item Se han integrado perturbaciones controladas en el entorno de juego terapéutico, lo que hace que sea variable y presente un desafío en el control motor y la capacidad de adaptación del paciente.
    \item Se ha diseñado una interfaz gráfica funcional y adaptable para que el terapeuta controle el desarrollo de la terapia y el modo de rehabilitación, permitiéndole visualizar el progreso del paciente y ajustar los parámetros del juego según su rendimiento.
    \item La arquitectura software de la plataforma es capaz de recibir datos de sensores y controlar actuadores para coordinar los estímulos visuales y físicos que el paciente experimente durante la terapia.
    \item Se ha creado una interfaz gráfica que permite definir los datos personales de un paciente y registrar su progreso después de cada sesión.
    \item Se ha evaluado la usabilidad de la plataforma a través de una encuesta de satisfacción a los usuarios.
\end{enumerate}\

Se concluye que los entornos gamificados incrementan significativamente la motivación de los usuarios durante el proceso de rehabilitación, según las encuestas de satisfacción realizadas a los usuarios que probaron la plataforma.
Los resultados graficados se encuentran en el Capítulo \ref{cap:capitulo5}.
Los usuarios disfrutan más cuando la terapia se desarrolla en un entorno dinámico e interactivo.
Además, la introducción de misiones secundarias dentro del juego ha demostrado una mejora del nivel de compromiso de los usuarios, ya que se sienten más involucrados en el proceso.

La personalización de las sesiones de rehabilitación, adaptadas a las necesidades de cada paciente, permite que el sistema se ajuste a su progreso y ofrece mejores resultados terapéuticos.
Según las encuestas, los usuarios se muestran más satisfechos con esta adaptación, porque les ofrece una terapia diseñada específicamente para sus capacidades.
Esto mejora la eficacia del tratamiento y deriva en una mayor adherencia.\\

Para finalizar, se proponen las siguientes mejoras que permiten la continuidad de este proyecto:
\begin{itemize}
    \item Mostrar el comienzo de la señal sinusoidal de tipo perturbación, en vez de un dibujar un triángulo, para que el usuario identifique el movimiento de forma más clara.
    \item Integrar una nueva señal que sea combinación de otras para introducir variabilidad a las señales de trayectoria y perturbación.
    \item Crear un programa que analice el rendimiento del paciente a partir de los datos terapéuticos almacenados. 
    \item Diseñar un nuevo juego enfocado en la mejora de la coordinación motora, ofreciendo una alternativa de rehabilitación.
    \item Evaluar la plataforma en pacientes con ictus para determinar su eficacia en un entorno sanitario.
\end{itemize}\