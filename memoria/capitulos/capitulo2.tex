\chapter{Objetivos}
\label{cap:capitulo2}

\begin{flushright}
\begin{minipage}[]{11cm}
\emph{Un objetivo bien definido es el punto de partida de todo logro.}\\
\end{minipage}\\

W. Clement Stone, \textit{Success Through a Positive Mental Attitude, 1959}\\
\end{flushright}

\vspace{1cm}
\setcounter{footnote}{0}

Después de haber establecido el marco contextual de este proyecto, continuo con la descripción del problema, los requisitos, las competencias, la metodología y el plan de trabajo empleado para su desempeño.\\

\section{Descripción del problema}
\label{sec:descripcion}

La recuperación motora tras un ictus requiere de terapias personalizadas y motivadoras que promuevan la participación activa del paciente.
Tal como señalan \cite{perales6a}, la ausencia de interfaces gráficas intuitivas y retroalimentación visual puede provocar una menor implicación del paciente durante el tratamiento, afectando de forma negativa los resultados terapéuticos.

En este contexto, el presente proyecto aborda dicha limitación mediante el diseño y desarrollo de un sistema de interacción gráfica que combina, por un lado, una plataforma de gamificación centrada en el paciente, con el objetivo de fomentar su participación y motivación, 
y por otro, una interfaz de visualización clínica centrada en el médico, que registre las métricas de desempeño y facilite el seguimiento de la evolución terapéutica.

\subsection{Objetivo general}
\label{sec:descripcion}

Diseñar una interfaz gráfica intuitiva y funcional para que un terapeuta controle y personalice las sesiones terapéuticas post-ictus, así como desarrollar un entorno gamificado que motive y facilite la participación activa del paciente durante su rehabilitación motora unilateral.
Con el fin de mejorar la eficacia de dichas sesiones y aumentar la adherencia del paciente al tratamiento, se monitorizan tanto el progreso como las métricas de ejecución del paciente.
De la misma manera, se almacenan los datos más relevantes durante la terapia para facilitar el análisis y la evaluación de los resultados.

\subsection{Objetivos específicos}
\label{sec:descripcion}

Con el fin de alcanzar esta meta, se han definido los siguientes objetivos específicos:

\begin{itemize}
    \item \textit{O1.} Desarrollar entornos gamificados que incluyan estímulos visuales y auditivos, niveles de dificultad y misiones particulares, adaptadas al perfil del paciente, con el objetivo mejorar su atención y motivación.
    \item \textit{O2.} Diseñar una interfaz gráfica intuitiva para que el terapeuta controle el desarrollo de la terapia y el modo de rehabilitación, permitiéndole visualizar el progreso del paciente y ajustar los parámetros del juego según su rendimiento.
    \item \textit{O3.} Establecer una arquitectura capaz de recibir datos de sensores y controlar actuadores para coordinar estímulos visuales y físicos que el paciente experimente durante la terapia, con el fin de crear un entorno inmersivo que potencie la eficacia del tratamiento.
    \item \textit{O4.} Crear una interfaz gráfica que permita definir los datos personales de un paciente y registrar su progreso depués de cada sesión, con el objetivo de personalizar y ajustar de forma automática los parámetros en futuras sesiones.
    \item \textit{O5.} Integrar perturbaciones controladas en el entorno de juego terapéutico, con el fin de introducir obstáculos que desafíen el control motor y la capacidad de adaptación del paciente, promoviendo una mayor atención, planificación motora y activación neuromuscular.
\end{itemize}\

\section{Requisitos}
\label{sec:requisitos}

A continuación, se enumeran los requisitos que se deben satisfacer:

\begin{itemize}
    \item Compatibilidad con la arquitectura del sistema robótico y su software de control, garantizando una integración fluida sin afectar al rendimiento del sistema.
    \item Diseño de una interfaz funcional y adaptable, que se ajuste a las necesidades terapéuticas del usuario impuestas por el equipo rehabilitador.
    \item Integración de señales de tipo perturbación controladas, para introducir variabilidad y desafíos en a terapia.
    \item Incorporación de elementos de gamificación, como niveles de dificultad y retroalimentación visual, que aumente la atención del paciente.
    \item Capacidad para adaptar la asistencia robótica, en función del esfuerzo realizado por el paciente, de forma progresiva.
    \item Registro de los datos terapéuticos generados durante las sesiones, para permitir su uso y análisis posterior.
    \item Validación del sistema mediante la valoración de los resultados obtenidos por los usuarios, considerando métricas objetivas y subjetivas.
\end{itemize}\

\section{Competencias}
\label{sec:competencias}

Durante la realización de este proyecto, se han puesto en práctica diversas competencias generales del grado y específicas del desarrollo de interfaces y sistemas dinámicos.

\subsection{Competencias generales}
\label{sec:competencias}

\begin{itemize}
    \item \textit{CG1:} Diseño software. Conocimiento de materias básicas y tecnologías, que le capacite para el aprendizaje de nuevos métodos y tecnologías, así como que le dote de una gran versatilidad para adaptarse a nuevas situaciones.
    \item \textit{CE15:} Arquitectura software para robots. Capacidad de diseñar y programar aplicaciones robóticas y sistemas inteligentes en red usando middlewares, mecanismos de comunicación y estándares propios del ámbito de la Ingeniería Robótica.
    \item \textit{CB2:} Planificación y sistemas cognitivos. Que los estudiantes sepan aplicar sus conocimientos a su trabajo o vocación de una forma profesional y posean las competencias que suelen demostrarse por medio de la elaboración y defensa de argumentos y la resolución de problemas dentro de su área de estudio.
    \item \textit{CB4:} Trabajo de Fin de Grado. Que los estudiantes puedan transmitir información, ideas, problemas y soluciones a un público tanto especializado como no especializado.
\end{itemize}\

\subsection{Competencias específicas}
\label{sec:competencias}

\begin{itemize}
    \item \textit{CE1.} Diseño e implementación de interfaces gráficas de usuario (GUI) con herramientas como Tkinter.
    \item \textit{CE2.} Desarrollo de aplicaciones interactivas con elementos de gamificación, aplicando principios de diseño centrado en el usuario.
    \item \textit{CE3.} Integración de interfaces con dispositivos físicos mediante comunicación en tiempo real basada en el Sistema Operativo Robótico (ROS).
    \item \textit{CE4.} Evaluación de la usabilidad y experiencia de usuario con métricas objetivas y percepciones.
\end{itemize}\

\section{Metodología}
\label{sec:metodologia}

Para llevar a cabo este proyecto, se ha optado por seguir el paradigma iterativo centrado en el usuario (User-Centered Design).
Previamente, se ha realizado una fase de análisis e investigación sobre el estado del arte y necesidades específicas para comprobar la viabilidad de su desarrollo.
Posteriormente, se procedió con el diseño del prototipo para validar el diseño gráfico y la navegación.
Más adelante, la realización de pruebas contínuas favoreció la incorporación de mejoras en el diseño visual, la lógica de juego y la retroalimentación.
Una vez se desarrolló una primera versión fiable, las interfaces se conectaron con el sistema de control del robot permitiendo crear un entorno cambiante y adaptable.
Por último, se llevó a cabo una evaluación de conformidad a los usuarios para comprobar la efectividad de la plataforma.

\section{Plan de trabajo}
\label{sec:plantrabajo}

El trabajo se ha organizado de forma progresiva y coordinada con el resto del equipo técnico.
Para ello, el proyecto se ha divido en las siguientes etapas:

\begin{enumerate}
    \item \textit{Investigación.} En esta fase inicial, se llevó a cabo un análisis detallado de las arquitecturas robóticas existentes en el ámbito de la rehabilitación, así como las propuestas lúdicas aplicadas en terapias interactivas. El fin de este estudio fue identificar funcionalidades relevantes y enfoques innovadores que servieran de inspiración para el diseño y desarrollo de una plataforma que combine control robótico, evaluación clínica y gamificación.
    \item \textit{Planteamiento del software.} Una vez se establecieron los objetivos y requisitos del proyecto, se procedió con el diseño conceptual de la plataforma, incluyendo un esquema de nodos, topics y flujos de datos en ROS 2. Este diseño permitió establecer de forma estructurada la arquitectura del sistema.
    \item \textit{Desarrollo de la parte gráfica.} A continuación, se desarrollaron las interfaces gráficas, orientadas al terapeuta, para permitir configurar parámetros clínicos, controlar el sistema robótico y almacenar los datos terapéuticos más relevantes por paciente. A través de continuas pruebas se optimizó tanto la aplicación como el uso de estas interfaces.
    \item \textit{Conexión entre el software y hardware.} Una vez consolidada una primera versión fiable del software, se continuó con la integración con el sistema robótico, incluyendo la sensorización del movimiento del paciente. Se validó la comunicación bidireccional entre los componentes software y hardware.
    \item \textit{Evaluación y conclusiones.} En esta etapa final, se realizaron pruebas de funcionamiento para evaluar la precisión del sistema, la generación de las señales de control y perturbación, y la usabilidad general de la plataforma. Los resultados obtenidos permitieron sacar conclusiones y recomendaciones para futuras mejoras.
\end{enumerate}\

Paralelamente, se ha ido completando la presente memoria a lo largo de este proceso.
Asimismo, se han realizado reuniones mensuales con el equipo para compartir los progresos de cada área, proponer mejoras y resolver cuestiones, trabajando de manera coordinada hacia un objetivo común.\\

A continuación, se procede a tratar las plataformas de desarrollo empleadas en este proyecto.
