\chapter{Objetivos}
\label{cap:capitulo2}

\begin{flushright}
\begin{minipage}[]{11cm}
\emph{Un objetivo bien definido es el punto de partida de todo logro.}\\
\end{minipage}\\

W. Clement Stone, \textit{Success Through a Positive Mental Attitude, 1959}\\
\end{flushright}

\vspace{1cm}
\setcounter{footnote}{0}

Después de haber establecido el marco contextual de este proyecto, continuo con la descripción del problema, los requisitos, las competencias, la metodología y el plan de trabajo empleado para su desempeño.\\

\section{Descripción del problema}
\label{sec:descripcion}

La recuperación motora tras un ictus requiere de terapias personalizadas y motivadoras que promuevan la participación activa del paciente.
Tal como señalan \cite{perales4b}, la ausencia de interfaces gráficas intuitivas y retroalimentación visual puede provocar una menor implicación del paciente durante el tratamiento, afectando de forma negativa los resultados terapéuticos.

En este contexto, el presente proyecto aborda dicha limitación mediante el diseño y desarrollo de un sistema de interacción gráfica que combina, por un lado, una plataforma de gamificación centrada en el paciente, con el objetivo de fomentar su participación y motivación, 
y por otro, una interfaz de visualización clínica centrada en el médico, que registre las métricas de desempeño y facilite el seguimiento de la evolución terapéutica.

\subsection{Objetivo general}
\label{sec:descripcion}

Diseñar un sistema que registre y almacene los datos clínicos más relevantes e implementar interfaces gráficas interactivas y entornos gamificados que mejoren la experiencia del usuario en terapias robóticas post-ictus.

\subsection{Objetivos específicos}
\label{sec:descripcion}

Con el fin de alcanzar esta meta, se han definido los siguientes objetivos específicos:

\begin{itemize}
    \item \textit{O1.} Desarrollar entornos gráficos gamificados que incluyan estimulos visuales, niveles de dificultad y misiones particulares que se adapten al perfil del usuario.
    \item \textit{O2.} Diseñar una interfaz gráfica intuitiva para la plataforma robótica, que permita al médico visualizar el progreso del paciente y ajustar los parámetros del juego en base a su rendimiento.
    \item \textit{O3.} Establecer sistemas de interacción visual conectados con los sensores y actuadores del robot para reflejar los movimientos del paciente y su desempeño.
    \item \textit{O4.} Crear una interfaz gráfica para definir los datos personales y el progreso del usuario después de cada sesión.
    \item \textit{O5.} Integrar perturbaciones controladas en el entorno de juego terapéutico, con el fin de introducir obstáculos que desafíen el control motor y la capacidad de adaptación del paciente, promoviendo así una mayor atención, planificación motora y activación neuromuscular.
\end{itemize}\

\section{Requisitos}
\label{sec:requisitos}

A continuación, se enumeran los requisitos que se deben satisfacer:

\begin{itemize}
    \item Compatibilidad con la arquitectura del sistema robótico y su software de control.
    \item Interfaz intuitiva y adaptable en base a las necesidades de cada usuario.
    \item Incorporación de elementos de gamificación como niveles o feedback visual.
    \item Conectividad con un actuador para adaptar la asistencia del robot según el esfuerzo del paciente.
    \item Posibilidad de registro y análisis de los datos terapéuticos para su posterior uso e investigación.
    \item Validación de uso mediante la valoración de los resultados obtenidos por los usuarios.
\end{itemize}\

\section{Competencias}
\label{sec:competencias}

Durante la realización de este proyecto, se han puesto en práctica diversas competencias generales del grado y específicas del desarrollo de interfaces y sistemas dinámicos.

\subsection{Competencias generales}
\label{sec:competencias}

\begin{itemize}
    \item \textit{CG1:} Conocimiento de materias básicas y tecnologías, que le capacite para el aprendizaje de nuevos métodos y tecnologías, así como que le dote de una gran versatilidad para adaptarse a nuevas situaciones.
    \item \textit{CG2:} Capacidad de resolver problemas con iniciativa, toma de decisiones, creatividad, y de comunicar y transmitir conocimientos, habilidades y destrezas, comprendiendo la responsabilidad ética y profesional de la actividad de la Ingeniería Robótica.
    \item \textit{CG3:} Redactar, representar e interpretar documentación legal, técnica, así como el manejo de especificaciones, reglamentos y normas de obligado cumplimiento en el ámbito de la robótica.
    \item \textit{CG4:} Capacidad de analizar y valorar el impacto social de los robots y el impacto medioambiental de las soluciones técnicas.
\end{itemize}\

\subsection{Competencias específicas}
\label{sec:competencias}

\begin{itemize}
    \item \textit{CE1.} Diseño e implementación de interfaces gráficas de usuario (GUI) con herramientas como Tkinter.
    \item \textit{CE2.} Desarrollo de aplicaciones interactivas con elementos de gamificación, aplicando principios de diseño centrado en el usuario.
    \item \textit{CE3.} Integración de interfaces con dispositivos físicos mediante comunicación en tiempo real basada en el Sistema Operativo Robótico (ROS\footnote{\url{https://www.ros.org/}}).
    \item \textit{CE4.} Evaluación de la usabilidad y experiencia de usuario con métricas objetivas y percepciones.
\end{itemize}\

\section{Metodología}
\label{sec:metodologia}

Para llevar a cabo este proyecto, se ha optado por seguir el paradigma iterativo centrado en el usuario (User-Centered Design).
Previamente, se ha realizado una fase de análisis e investigación sobre el estado del arte y necesidades específicas para comprobar la viabilidad de su desarrollo.
Posteriormente, se procedió con el diseño del prototipo para validar el diseño gráfico y la navegación.
Más adelante, la realización de pruebas contínuas favoreció la incorporación de mejoras en el diseño visual, la lógica de juego y la retroalimentación.
Una vez se desarrolló una primera versión fiable, las interfaces se conectaron con el sistema de control del robot permitiendo crear un entorno cambiante y adaptable.
Por último, se llevó a cabo una evaluación de conformidad a los usuarios para comprobar la efectividad de la plataforma.

\section{Plan de trabajo}
\label{sec:plantrabajo}

El trabajo se ha organizado de forma progresiva y coordinada con el resto del equipo técnico.
Para ello, el proyecto se ha divido en las siguientes etapas:

\begin{enumerate}
    \item \textit{Investigación.} En esta fase inicial, se llevó a cabo un análisis detallado de plataformas robóticas existentes en el ámbito de la rehabilitación, como el dispositivo Armeo Spring\footnote{\url{https://www.comunidad.madrid/noticias/2019/04/16/comunidad-incorpora-robot-rehabilitacion-hospital-guadarrama}}, entre otros. El fin de este estudio fue identificar funcionalidades relevantes y enfoques innovadores que pudieran servir de inspiración para el diseño y desarrollo de la nueva plataforma.
    \item \textit{Planteamiento del software.} Una vez se establecieron los objetivos y requisitos del proyecto, se procedió con el diseño de un esquema de nodos y topics que permitió estructurar de forma organizada la arquitectura de la plataforma.
    \item \textit{Desarrollo de la parte gráfica.} Después de definir la arquitectura del sistema, se inició el desarrollo de las interfaces gráficas. A través de continuas pruebas se fueron optimizando tanto su aplicación como su uso.
    \item \textit{Conexión entre el software y hardware.} Una vez consolidada una primera versión fiable del software, se continuó con la integración de las interfaces con el sistema de sensonorización del robot.
    \item \textit{Evaluación y conclusiones.} En esta etapa final, se evaluó la usabilidad y se validaron los controles de forma técnica.
\end{enumerate}\

Paralelamente, se ha ido completando la presente memoria a lo largo de este proceso.
Asimismo, se han realizado reuniones mensuales con el equipo para compartir los progresos de cada área, proponer mejoras y resolver cuestiones, trabajando de manera coordinada hacia un objetivo común.\\

A continuación, se procede a tratar las plataformas de desarrollo empleadas en este proyecto.
